%%%%%%%%%%%%%%%%%%%%%%%%%%%%%%%%%%%%%%%%%%%%%%%%%%%%%%%%%%%%%%%%%%%%%%%%%%
%
%    phase1-GO.tex  (use only for General Observer and Snapshot proposals; 
%                      use phase1-AR.tex for Archival Research and
%                      Theory proposals use phase1-DD.tex for GO/DD proposals).
%
%    HUBBLE SPACE TELESCOPE
%    PHASE I OBSERVING PROPOSAL TEMPLATE 
%     FOR CYCLE 23 (2015)
%
%    Version 1.0, January 07, 2015.
%
%    Guidelines and assistance
%    =========================
%     Cycle 23 Announcement Web Page:
%
%         http://www.stsci.edu/hst/proposing/docs/cycle23announce 
%
%    Please contact the STScI Help Desk if you need assistance with any
%    aspect of proposing for and using HST. Either send e-mail to
%    help@stsci.edu, or call 1-800-544-8125; from outside the United
%    States, call [1] 410-338-1082.
%
%
%%%%%%%%%%%%%%%%%%%%%%%%%%%%%%%%%%%%%%%%%%%%%%%%%%%%%%%%%%%%%%%%%%%%%%%%%%%

% The template begins here. Please do not modify the font size from 12 point.

\documentclass[12pt]{article}
\usepackage{phase1}

%%%%%%%%%%%%%%%%%%%%%%%%%%%%%%%%%%%%%%%%%%%%%%%%%%%%%%%%%%%%%
%  PREAMBLE: sets up compiler modes, loads packages, defines macros, etc
%  Steve Rodney, 2012
%%%%%%%%%%%%%%%%%%%%%%%%%%%%%%%%%%%%%%%%%%%%%%%%%%%%%%%%%%%%%


%%%%%%%%%%%%%%%%%%%%
% COMPILER MODES
%%%%%%%%%%%%%%%%%%%%

% Changetext mode : highlight modified text in bold blue font
\newif{\ifchangetext}
\changetextfalse

%% Read in the -options.tex file (generated by the Makefile)
%%  to set the compile-mode options
\InputIfFileExists{\jobname-options}


%%%%%%%%%%%%%%%%%%%%%%%%%%%%%%%
% changetext  mode settings
%%%%%%%%%%%%%%%%%%%%%%%%%%%%%%%
\ifchangetext
  % Changed text is highlighted in bold, blue font 
  \newcommand{\change}[1]{{\bf \textcolor{blue}{#1}}}
\else
  % Changed text is indistinguishable
  \newcommand{\change}[1]{#1}
\fi


%%%%%%%%%%%%%%%%%%%%
% PACKAGES INCLUDED
%%%%%%%%%%%%%%%%%%%%
%\usepackage{wrapfig}
\usepackage{deluxetable} % deluxetable(*) environments
\usepackage{caption} % explicit figure captions
\usepackage{graphicx} % includegraphics commands
\usepackage{journalnames} % Astro Journal abbreviations
\usepackage{natbib}   % reference citations and bibliography
%\usepackage{amsmath}  % extended math symbols
\usepackage{verbatim} % verbatim text formatting
\usepackage{enumerate}% enumerated lists
%\usepackage{amssymb}  % extended symbols lib
\usepackage{url}      % url text formatting
\usepackage[usenames]{color}  % colored text
%\usepackage{multirow}  % muti-row table cells
%\usepackage{amsmath}  % extended equations lib (split)
%\usepackage{mathrsfs} % extended math fonts (mathscr)
%\usepackage{paralist} % inline enumeration (for Table ref lists)
%\usepackage{authblk}
\usepackage{multicol}
%\usepackage{subfig} % subfloats with independent captions
\usepackage{subcaption} % subfloats with independent captions
\usepackage{setspace} % switch from double to single spacing
\usepackage[none]{hyphenat} % Suppress the hyphenating
\usepackage{ulem} % for some underlining.
\usepackage{wrapfig}

%%%%%%%%%%%%%%%%%%%%%%%%%%%%%%%%%%%%%%%%%%%%%%%%%%%
% Redefine thebibliography for two-column formatting
% with the ``References'' title spanning the page
% http://tex.stackexchange.com/questions/20758/bibliography-in-two-columns-section-title-in-one
%   2013.02.22 : doesn't work with HST proposal sty files...?
%%%%%%%%%%%%%%%%%%%%%%%%%%%%%%%%%%%%%%%%%%%%%%%%%%%
\makeatletter
\renewenvironment{thebibliography}[1]
     {
  \begin{multicols}{2}[\section*{\refname}]%
      \@mkboth{\MakeUppercase\refname}{\MakeUppercase\refname}%
      \list{\@biblabel{\@arabic\c@enumiv}}%
           {\settowidth\labelwidth{\@biblabel{#1}}%
            \leftmargin\labelwidth
            \advance\leftmargin\labelsep
            \@openbib@code
            \usecounter{enumiv}%
            \let\p@enumiv\@empty
            \renewcommand\theenumiv{\@arabic\c@enumiv}}%
      \sloppy
      \clubpenalty4000
      \@clubpenalty \clubpenalty
      \widowpenalty4000%
      \sfcode`\.\@m}
     {\def\@noitemerr
       {\@latex@warning{Empty `thebibliography' environment}}%
      \endlist\end{multicols}}
\makeatother

%%%%%%%%%%%%%%%%%%%%%%%%%%%%%%%%%%%%%%%%%%%%%%%%%%%
% PDF mode settings : Auto-select eps or pdf figures 
% based  on the compiler used (i.e. latex vs pdflatex)
%%%%%%%%%%%%%%%%%%%%%%%%%%%%%%%%%%%%%%%%%%%%%%%%%%%
\ifx\pdfoutput\undefined
  \pdffalse
  \DeclareGraphicsExtensions{.eps,.ps}
\else
  \ifnum\pdfoutput=1
    % \pdftrue
    \DeclareGraphicsExtensions{.pdf,.png,.jpg}
  \else
    % \pdffalse
    \DeclareGraphicsExtensions{.eps,.ps}
  \fi
\fi


%%%%%%%%%%%%%%%%%%%%%%%%%%%%%%%
% AUTHOR-DEFINED MACROS
%%%%%%%%%%%%%%%%%%%%%%%%%%%%%%%
% General purpose usefulness:
\newcommand{\etal}{{et al.~}}                                             
\def\eg{{e.g.}}
\def\ie{{i.e.}}
\def\etc{{etc.}}
\newcommand{\lta}{\lesssim}                                               
\newcommand{\gta}{\gtrsim}                                                
\newcommand{\gt}{\gtsim}
%\newcommand{\kms}{\,\rm km\,s^{-1}}                                       

% Cosmology:
\def\Om{\ensuremath{\Omega_{\rm m}}}
\def\Ot{\ensuremath{\Omega_{\rm tot}}}
\def\Ob{\ensuremath{\Omega_{\rm b}}}
\def\OL{\ensuremath{\Omega_{\Lambda}}}
\def\Ok{\ensuremath{\Omega_{\rm k}}}
\def\om{\ensuremath{\omega_{\rm m}}}
\def\ob{\ensuremath{\omega_{\rm b}}}
\def\wo{\ensuremath{w_0}}
\def\wa{\ensuremath{w_{\rm a}}}
\def\lcdm{$\Lambda$CDM}
\def\LCDM{$\Lambda$CDM}
\def\wcdm{$w$CDM}
\def\Ho{\ensuremath{H_0}}
\def\DA{\ensuremath{D_A}}
\def\DL{\ensuremath{D_L}}

% Astronomy:
\def\arcsec{\ensuremath{^{\prime\prime}}} 
\def\kms{\ensuremath{{\rm km s}^{-1}}}
\def\hgpcq{\mbox{$h^{-3}$Gpc$^3$}}
\def\hmpcq{\mbox{$h^{-3}$Mpc$^3$}}
\def\perhmpcq{\mbox{$h^{3}$Mpc$^{-3}$}}
\def\hmpc{\mbox{$h^{-1}$Mpc}}
\def\hmpci{\mbox{$h$\,Mpc$^{-1}$}}
\def\mpc{\mbox{Mpc}}
\def\mpci{\mbox{Mpc$^{-1}$}}
\def\mpcq{\mbox{Mpc$^{-3}$}}
\def\Msun{\mbox{M$_{\odot}$}}
\def\Av{\mbox{$A_V$}}
\def\Rv{\mbox{$R_V$}}

% Supernovae : 
\newcommand{\tomas}{HFF14Tom}
\newcommand{\primo}{SN~Primo}
\newcommand{\CCSN}{CC\,SN}
\newcommand{\CCSNe}{CC\,SN}
\newcommand{\TNSN}{TN\,SN}
\newcommand{\TNSNe}{TN\,SNe}
\newcommand{\SNIa}{SN\,Ia}
\newcommand{\SNe}{SNe}
\newcommand{\SNeIa}{SN\,Ia}
\newcommand{\SNRz}{SNR($z$)}
\def\Mch{\mbox{M$_{\rm Ch}$}}
\def\Ni{\ensuremath{^{56}\mbox{Ni}}}
\newcommand{\dmfifteen}{\ensuremath{\Delta\mbox{m}_{15}}}
\newcommand{\deltamfifteen}{\ensuremath{\Delta\mbox{m}_{15}}}

% Missions:
\def\HST{{\it HST}}
\def\Hubble{{\it Hubble}}
\def\Hubbles{{\it Hubble's}}
\def\Spitzer{{\it Spitzer}}
\def\Chandra{{\it Chandra}}
\def\Herschel{{\it Herschel}}
\def\XMM{{\it XMM}}

% Institutions
\newcommand{\JHU}{Department of Physics and Astronomy, The Johns Hopkins University, Baltimore, MD 21218.}
\newcommand{\elsewhere}{St. Elsewhere University.}
\newcommand{\STScI}{Space Telescope Science Institute, Baltimore, MD 21218.}
\newcommand{\Berkeley}{Department of Astronomy, University of California, Berkeley, CA 94720-3411.}
\newcommand{\Riverside}{Department of Physics and Astronomy, University of California, Riverside, CA 92521.}
\newcommand{\WKU}{Department of Physics, Western Kentucky University, Bowling Green, KY 42101.}
\newcommand{\Copenhagen}{Dark Cosmology Centre, Niels Bohr Institute, University of Copenhagen, Juliane Maries Vej 30, DK-2100 Copenhagen, Denmark.}
\newcommand{\Arizona}{Department of Astronomy, University of Arizona, Tucson, AZ 85721.}
\newcommand{\SantaCruz}{Department of Astronomy and Astrophysics, University of California, Santa Cruz, CA 92064.}
\newcommand{\NotreDame}{Department of Physics, University of Notre Dame, Notre Dame, IN 46556.}
\newcommand{\TelAviv}{Department of Astrophysics, Tel Aviv University, 69978 Tel Aviv, Israel.}
\newcommand{\Rutgers}{Department of Physics and Astronomy, Rutgers, The State University of New Jersey, Piscataway, NJ 08854.}
\newcommand{\CfA}{Harvard/Smithsonian Center for Astrophysics, Cambridge, MA 02138.}
\newcommand{\Minnesota}{Department of Astronomy, University of Minnesota, 116 Church Street SE, Minneapolis, MN 55455, USA.}


%% Define a new 'leo' style for the url package 
%% that will use a smaller font.
% \def\UrlFont{\tt}
%\makeatletter
%\def\url@leostyle{%
%  \@ifundefined{selectfont}{\def\UrlFont{\sf}}{\def\UrlFont{\small\ttfamily}}}
%\makeatother
%%% Now actually use the newly defined style.
%\urlstyle{leo}


%%%%%%%%%%%%%%%%%%%%%%%%%%%%%%%
% Page Setup 
%%%%%%%%%%%%%%%%%%%%%%%%%%%%%%%
%  \renewcommand{\topfraction}{0.9}
%  \renewcommand{\bottomfraction}{0.9}
%  \renewcommand{\textfraction}{0.1}
%  \renewcommand{\floatpagefraction}{0.9}
%  \renewcommand{\dbltopfraction}{0.9}
%  \renewcommand{\dblfloatpagefraction}{0.9}



%%%%%%%%%%%%%%%%%%%%%%%%%%%%%%%%%%%%
%% Figure placement shortcuts    
%%%%%%%%%%%%%%%%%%%%%%%%%%%%%%%%%%%%

\newcommand{\figcaption}[1]{
\caption{{#1}}
}

\newenvironment{inlinefigure}{
\def\@captype{figure}
\noindent\begin{minipage}{0.999\linewidth}\begin{center}}
{\end{center}\end{minipage}\smallskip}

\newcommand{\insertfigwide}[2] {
\begin{figure*}
\begin{center}
\resizebox{\textwidth}{!}{\includegraphics{{#1}}}
\caption{{#2}}
\end{center}
\end{figure*}
}

\newcommand{\insertfiginline}[2] {
\begin{inlinefigure}
\begin{center}
\resizebox{\columnwidth}{!}{\includegraphics{{#1}}}
\figcaption{{#2}}
\end{center}
\end{inlinefigure}
}

\newcommand{\insertfigfloat}[2] {
\begin{figure}
\begin{center}
\resizebox{\columnwidth}{!}{\includegraphics{{#1}}}
\figcaption{{#2}}
\end{center}
\end{figure}
}

\newcommand{\insertfig}[2] {
\insertfigfloat{{#1}}{{#2}}
}

\newcommand{\insertfigdouble}[3] {
\begin{figure*}
\begin{center}
\resizebox{0.45\textwidth}{!}{\includegraphics{{#1}}}
\resizebox{0.45\textwidth}{!}{\includegraphics{{#2}}}
\figcaption{{#3}}
\end{center}
\end{figure*}
}


\newcommand{\insertfigend}[2] {
\newpage
%\vspace*{1.0in}
\begin{inlinefigure}
\begin{center}
\resizebox{\textwidth}{!}{\includegraphics{{#1}}}
\figcaption{{#2}}
\end{center}
\end{inlinefigure}
}


\begin{document}

%   1. SCIENTIFIC JUSTIFICATION
%       (see Section 9.1 of the Call for Proposals)
%
%
\justification          % Do not delete this command.
% Enter your scientific justification here. 

Over the past decade, the deep \HST\ Treasury surveys (GOODS, CANDELS,
CLASH) have proven to be the most effective programs for locating and
studying Supernovae (SNe) at high redshift.  These surveys have all
enabled ``piggyback'' SN searches (hereafter, the HST-SN surveys),
which have collectively accumulated scores of SN detections that reach
to uniquely high redshifts
\citep{Riess:2007,Dahlen:2008,Graur:2014a,Rodney:2014}.  Continuing in
this line of highly successful deep \HST\ surveys, the Hubble Frontier
Fields (HFF) director's discretionary program (PI:Lotz) now provides a
powerful new tool for the discovery of high-$z$ SNe.  What sets the
HFF survey apart from the previous HST-SN surveys is the unique depth
of each visit; with $\sim$4 orbits per filter per epoch, we can reach
$m_{lim,3\sigma}(F160W)\approx27.9 (AB)$, nearly 1 mag deeper than
CANDELS/CLASH per epoch. Gravitational lensing in the prime fields
also magnifies SN fluxes by factor $\mu\gtrsim2$, making it possible
to detect unique background events at extreme redshifts. 

In Cycle 21 the TAC awarded us 60 orbits and 15 non-disruptive ToO
triggers over 3 cycles to use in follow-up observations of SNe
discovered in HFF imaging and supplementary surveys. We are on pace to
expend all of these orbits in the first 2 years, primarily due to the
unexpected (and unprecedented) discovery of a multiply-imaged SN,
gravitationally lensed into an Einstein Cross
(Figure~\ref{fig:refsdal}, \citealt{Kelly:2015}).  {\bf We propose to
  extend our FrontierSN program into Cycle 23, so we can continue to
  provide the necessary follow-up observations to maximize the
  scientific return of this rich Frontier Frields SN sample.}

\smallskip
\noindent {\bf The FrontierSN Sample:} 
Through the first two years of the HFF program, our team has searched
for SNe in all imaging of the Frontier Fields clusters and parallel
fields.  We have also searched for SNe in the 10 complementary cluster
and parallel fields observed in the Grism Lens Amplified Survey from
Space (GLASS, PI:Treu, PID:13495), working jointly with the GLASS
team.  In the first two years of the FrontierSN program, we have
discovered 37 explosive transients reaching to $z=1.5$, which are
mostly normal Type Ia or Core Collapse SNe.  The bulk of our follow-up
observations have been expended on a small subset of unique and
peculiar objects behind the cluster, which are providing powerful new
tools for testing and improving cluster mass models.


Our primary science goal with
this sample is to measure \SNIa\ rates to $z\sim2.5$, yielding
improved constraints on \SNIa\ progenitor models.  These SNe will also
provide a check on possible evolution of the \SNIa\
population \citep{Riess:2006}, as well as unique tests for the mass
models of the Frontier Field clusters \citep{Riehm:2011}. Moreover,
this sample will extend the measurement of \CCSN\ rates to $z>1$,
giving us valuable direct measurements of the star formation rate
density \citep{Dahlen:2012}.

\smallskip
\noindent {\bf Follow-up Needed:} 
Although the Frontier Field survey enables the initial discovery of
SNe for ``free'', in order to reach any of these science goals we must
also be able to accurately distinguish \SNIa\ from \CCSNe, especially
at $z>1$.  Alas, the Frontier Field survey will not deliver coincident
optical$-$IR colors, which dramatically improve the discrimination
between red \SNeIa\ and blue \CCSNe.  The narrow observing windows
will also guarantee that high-z SN light curves are only partially
sampled by the Frontier survey epochs.  A well-sampled light curve is
necessary to enable
\SNIa\ light curve fitting for luminosity distance determination. 
Therefore, {\it we propose to obtain non-disruptive ToO imaging
observations for $\sim$13 SN targets of interest.}  As detailed below
in the Description of Observations, our request for 20 ToO orbits in
Cycle 23 will provide the colors for initial classifications, the
post-Frontier epochs for full light curve sampling, and grism
spectroscopy or medium band IR
imaging for classifications and redshifts of the unique strongly
lensed SN sample.


%%%%%%%%%%%%%%%%%%%%%%%%%%%%%%%%%%%%%%%%%%%%%%%%%%%%%%%%%%%%%%%%%%%%%%%%%%%

%   2. DESCRIPTION OF THE OBSERVATIONS
%       (see Section 9.2 of the Call for Proposals)
%
%
\describeobservations   % Do not delete this command.
% Enter your observing description here.

%%%%%%%%%%%%%%%%%%%%%%%%%%%%%%%%%%%%%%%%%%%%%%%%%%%%%%%%%%%%%%%%%%%%%%%%%%%

%   3. SPECIAL REQUIREMENTS
%       (see Section 9.3 of the Call for Proposals)
%
%
\specialreq             % Do not delete this command.
% Justify your special requirements here, if any.

%%%%%%%%%%%%%%%%%%%%%%%%%%%%%%%%%%%%%%%%%%%%%%%%%%%%%%%%%%%%%%%%%%%%%%%%%%%

%   4. COORDINATED OBSERVATIONS
%       (see Section 9.4 of the Call for Proposals)
%
%
\coordinatedobs          % Do not delete this command.
% Enter your coordinated observing plans here, if any.

%%%%%%%%%%%%%%%%%%%%%%%%%%%%%%%%%%%%%%%%%%%%%%%%%%%%%%%%%%%%%%%%%%%%%%%%%%%

%   5. JUSTIFY DUPLICATIONS
%       (see Section 9.5 of the Call for Proposals)
%
%
\duplications           % Do not delete this command.
% Enter your duplication justifications here, if any.

%%%%%%%%%%%%%%%%%%%%%%%%%%%%%%%%%%%%%%%%%%%%%%%%%%%%%%%%%%%%%%%%%%%%%%%%%%%


%   6. PAST HST USAGE
%       (see Section 9.8 of the Call for Proposals)
%
%        List here the program numbers and data status for all accepted GO/AR/SNAP 
%        programs of the PI in at least the last four HST Cycles. Include a list of refereed publications 
%        resulting from these programs.       
%
%       Note that the description of past HST usage  DOES NOT count against the page limits of the proposal.
%
\pasthstusage  % Do not delete this command.

    List here the program numbers and data status for all accepted GO/AR/SNAP Programs of the PI in the last four HST cycles. Include a list of refereed publications resulting from these programs.
       
    Note that the description of past HST usage DOES NOT count against the page limits of the proposal.

% List here the program numbers and data status for all accepted GO/AR/SNAP
% programs of the PI in at least the last four HST Cycles. Include a list of refereed
% publications resulting from these programs.

%%%%%%%%%%%%%%%%%%%%%%%%%%%%%%%%%%%%%%%%%%%%%%%%%%%%%%%%%%%%%%%%%%%%%%%%%%%

\end{document}          % End of proposal. Do not delete this line.
                        % Everything after this command is ignored.

